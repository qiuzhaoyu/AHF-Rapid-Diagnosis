\section{Introduction}\label{sec:introduction}
\subsection{Background}
Cardiovascular disease (CVD), including heart failure (HF) and stroke, has become the leading cause of death and disability worldwide \cite{virani2020heart,roth2020global,mensah2019global,boorsma2020congestion}. The rising age-standardized rate of CVD is especially evident in almost all non-high-income countries \cite{roth2020global}. Heart failure, particularly in its acute form (AHF), represents a critical stage in CVD, characterized by the heart's impaired ability to contract and stretch. AHF is a leading cause of emergency hospital admissions, especially among the elderly population \cite{sinnenberg2020acute,arrigo2020acute}. It accounts for over 26 million hospital admissions annually, with a mortality rate ranging from 20-30\% \cite{chapman2019clinical}. In the management of AHF, emergency procedures such as ambulance response time and door-to-balloon time are crucial, as reducing these intervals can significantly decrease patient mortality and morbidity \cite{victor2012door,fan2021effects}. For instance, in myocardial infarction—a common cause of AHF—reducing ambulance response time by 15.3 minutes and door-to-balloon time by 36 minutes has been shown to decrease hospital stays by 6.3 days, lower mortality by 12.33\%, and reduce rehospitalization rates by 19.69\% \cite{fan2021effects}. Consequently, optimizing these procedures, particularly by shortening door-to-balloon time and conducting rapid diagnostics such as auscultation during ambulance transport, can greatly enhance survival rates in AHF patients. However, traditional diagnostic methods for AHF, including auscultation, require further optimization.
While auscultation is a key rapid diagnostic tool, its effectiveness in ambulance scenarios requires additional quantitative evaluation to ensure it meets the urgent demands of modern emergency care.

Traditional AHF diagnosis relies heavily on clinical biochemical trials, and there is a significant lack of convenient and accurate diagnostic methods, particularly for the 20\% of patients with no history of chronic heart failure (CHF). According to the European Society of Cardiology's AHF First Aid Guidelines, the diagnostic criteria for AHF encompass clinical evaluation, electrocardiogram (ECG), chest x-ray, imaging techniques, laboratory tests, and echocardiography \cite{nieminen2005task}. Diagnosis typically hinges on the patient's history and physical examination, supplemented by physical tests such as ECG, echocardiography, and biochemical markers like brain natriuretic peptide (BNP) and NT-proBNP. However, current methods are time-consuming; for example, mainstream NT-proBNP and BNP biochemical detection equipment reports waiting times of 9-16 minutes for BNP results and 11-21 minutes for NT-proBNP results \cite{lewis2020bnp}. Additionally, echocardiography, which provides crucial information on the ejection fraction for preoperative evaluation of AHF, requires about 20-30 minutes for a single examination, including approximately five minutes for equipment setup \cite{menon2022echocardiography}. These time requirements highlight the need for more rapid and accessible diagnostic tools, especially in emergency settings where timely decision-making is critical.

Auscultation, as part of the clinical evaluation, is an effective means of diagnosing heart failure and is recommended by the European Society of Cardiology as a class-1 method \cite{nieminen2005task}. Heart sounds are produced by the mechanical motion of the heart's dynamic system, comprising various mechanical vibrations caused by blood movement within the cardiovascular system. These heart sounds contain a wealth of information regarding the physiology of the cardiovascular system and serve as a crucial auditory representation of the heart's contractile and stretch functions \cite{johnston2007third,wynne2001clinical,boorsma2020congestion}. Normal heart sounds consist of four distinct tones, known as the first heart sound (S1), second heart sound (S2), third heart sound (S3), and fourth heart sound (S4), occurring in sequence during the cardiac cycle. The mitral valve, aortic valve, pulmonary valve, and tricuspid valve are the four most commonly used auscultation areas. The tricuspid valve area, in particular, is crucial for detecting right heart failure. Unlike ECG, which records electrical activity, heart sounds reflect the ventricle's ability to pump blood, providing potentially richer pathological information within a single cardiac cycle. Abnormal heart sounds generally exhibit more background noise, greater fluctuations in S1 amplitude, and more high-frequency details compared to normal heart sounds, making auscultation a valuable tool for early detection of heart dysfunction.
\subsection{Related works}
The current research on digital auscultation technology encompasses aspects such as datasets, signal processing, and diagnostic models.

In terms of datasets, the most commonly utilized dataset is the one established by PhysioNet for the Heart Sound Classification Challenge in 2016 \cite{clifford2016classification}, including 665 abnormal heart sounds and 2575 normal heart sounds. Yaseen et al. \cite{son2018classification} provided a five-classification dataset of Aortic Stenosis (AS), Mitral Regurgitation (MR), Mitral Stenosis (MS), Mitral Valve Prolapse (MVP) and Normal (N), with 200 cases of each type. Nonetheless, current datasets may have limitations in annotation richness, which can hinder the development of algorithms aimed at diagnosing specific medical conditions. Therefore, to develop rapid diagnostic models for AHF, additional efforts are needed to refine the data and standardize inclusion criteria.

In the realm of pre-processing and feature extraction for heart sounds, various techniques have been developed to analyze these signals effectively. For instance, Vepa's research \cite{vepa2009classification} utilized Short-Time Fourier Transform (STFT) and Discrete Wavelet Transform (DWT) to process heart sound signals. Wu et al. \cite{wu2010hidden} extracted Mel-Frequency Cepstral Coefficients (MFCC) components of heart sound signals based on the hidden Markov model (HMM). Techniques such as MFCC-based feature extraction have proven to be highly successful and widely adopted in heart sound processing. While there are classification algorithms capable of handling short-duration phonocardiogram (PCG) segments, often as short as 2 seconds with majority voting for final output, many existing methods are optimized for longer signal durations, which may not always be ideal for the rapid diagnosis required in acute heart failure (AHF) scenarios. This highlights the need for further optimization of these techniques to ensure they are suitable for the time-sensitive nature of AHF diagnosis.

In terms of diagnostic models, Rubin \cite{rubin2016classifying} used a convolutional neural network (CNN) for heart sound signal classification based on time-frequency characteristics. Arora et al. \cite{arora2021transfer,xiang2023research} performed transfer learning of heart sound signals using models like VIZ, MobileNet, Xception, VGG, ResNet, DenseNet, and Inception. Scholars such as Li and Shuvo \cite{li2021lightweight,shuvo2021cardioxnet} have developed end-to-end lightweight neural networks for clinical mobile devices. Chen et al. \cite{chen2023robust} introduced an attention mechanism in the field of heart sound classification and used focal loss as the loss function to address the issue of data imbalance. Existing models are often too large and computationally intensive for the rapid diagnosis of AHF. Therefore, we have developed lightweight models and introduced different auscultation strategies for various clinical scenarios.


To address the challenges of signal processing and diagnostic model in AHF auscultation, our main contributions are:

\begin{itemize}
\item  An auscultation dataset has been established, containing 2999  recordings from heart failure patients, aiming at addressing the challenges in heart failure auscultation. This dataset encompasses comprehensive information, including diagnostic results, collection area annotations, medical history records, and annotations related to BNP and NT-proBNP. Additionally, this dataset has undergone a rigorous ethical review process and is publicly accessible on  \href{https://github.com/qiuzhaoyu/AHF-Rapid-Diagnosis}{https://github.com/qiuzhaoyu/AHF-Rapid-Diagnosis}.
\item A wavelet denoising algorithm and a lightweight DenseHF-Net have been developed for short-duration auscultation signals, aiming at rapid diagnosis of acute heart failure. The denoising algorithm proposed in this paper has improved the average signal-to-noise ratio to 7.8 dB. DenseHF-Net has only 0.33M parameters and is easily ported to low-computation scenarios such as mobile terminals. 
\item We have introduced two auscultation strategies: multi-region fusion auscultation and mitral valve auscultation. Multi-region fusion auscultation is designed for scenarios where long-time auscultation is possible, such as monitoring situations. It utilizes the three most crucial regions, namely, the mitral valve region, aortic valve region, and pulmonic valve region. On the other hand, mitral valve auscultation is specifically designed for rapid diagnosis in cases of heart failure emergencies, focusing solely on the mitral valve auscultation region. It completes the diagnosis within 15 seconds with an accuracy of 92.60\%.
\end{itemize}

% The remainder of this paper is arranged as follows:
% Paragraph.\ref{Materials and Methods} describes data denoising, feature extraction and models building. Paragraph.\ref{Results} describes the results of Paragraph.\ref{Materials and Methods}. Paragraph.\ref{Discussion} discusses the experimental results and compares them with other research. Paragraph.\ref{Conclusion} summarises the results and shortcomings of this paper, and suggests future research.